%% \begin{figure}[h]


%% \includegraphics[width=3.4in]{/home/fh/sci/lab/aniso_netw/data/plos_img/217066f4.pdf} %?? R!!
%% \includegraphics[width=1.7in]{/home/fh/load/frequency_try_tanfit.pdf}

%% \includegraphics[width=3.4in]{/home/fh/sci/lab/aniso_netw/data/plos_img/5580e5db.pdf} %?? R!!
%% \includegraphics[width=1.7in]{/home/fh/load/frequency_try_tanfit.pdf}
%% \includegraphics[width=3.4in]{/home/fh/sci/lab/aniso_netw/data/plos_img/7c211d09.pdf} %?? R!
%% \includegraphics[width=1.7in]{/home/fh/load/frequency_try_tanfit.pdf}

%% \includegraphics[width=5.44in]{/home/fh/sci/lab/aniso_netw/data/plos_img/33e7fb1d.pdf}
%% \caption{{\bf LOOK AT DIMENSIONS (PLOSMAIN) BEFORE MAKING FIGURE Strong influence of anisotropy on the neuro cluster counts}
%% A-C: Important: No 3 neuron motifs with 6 edges in tuned anisotropic
%% graphs registered}
%% \label{fig_cluster}%?? proper label s
%% \end{figure}



%\documentclass[crop=true,border=0,convert={density=600}]{standalone}
\documentclass[crop=true,border=0]{standalone}

\usepackage{graphicx}
\usepackage{xcolor}
\usepackage{overpic}
\usepackage{tgheros}
\usepackage[export]{adjustbox}
\usepackage{calc}

\renewcommand*{\familydefault}{\sfdefault}


\begin{document}

\def\xin{5}
\def\yin{87.5}
\def\wx{3.4in}
\def\wy{1.7in}

\Large \bfseries

\begin{tabular}{cc} 

  \begin{overpic}[width=\wx]%
    {/home/fh/sci/lab/aniso_netw/data/plos_img/217066f4.pdf} %?? 
    %% \put(\xin,\yin){\fboxsep=3pt\colorbox{white}{A}}
  \end{overpic}

  \begin{overpic}[width=\wy]%
    {/home/fh/load/frequency_try_tanfit.pdf}
    %% \put(\xin,\yin){\fboxsep=3pt\colorbox{white}{B}}
  \end{overpic}
  \\

  \begin{overpic}[width=\wx]%
    {/home/fh/sci/lab/aniso_netw/data/plos_img/5580e5db.pdf} %??
    %% \put(\xin,\yin){\fboxsep=3pt\colorbox{white}{A}}
  \end{overpic}
  
  \begin{overpic}[width=\wy]%
    {/home/fh/load/frequency_try_tanfit.pdf}
    %% \put(\xin,\yin){\fboxsep=3pt\colorbox{white}{B}}
  \end{overpic}

  \\
  
  \begin{overpic}[width=\wx]%
    {/home/fh/sci/lab/aniso_netw/data/plos_img/7c211d09.pdf} %?? R!
    %% \put(\xin,\yin){\fboxsep=3pt\colorbox{white}{A}}
  \end{overpic}

  \begin{overpic}[width=\wy]%
    {/home/fh/load/frequency_try_tanfit.pdf}
    %% \put(\xin,\yin){\fboxsep=3pt\colorbox{white}{B}}
  \end{overpic}

  \\

  \multicolumn{2}{c}{
      \begin{overpic}[width=\wx+\wy]%
        {/home/fh/sci/lab/aniso_netw/data/plos_img/33e7fb1d.pdf} %?? 
        %% \put(\xin,\yin){\fboxsep=3pt\colorbox{white}{A}}
      \end{overpic}
  }
  
  %% \\

  %% \begin{overpic}[width=\w-0.21in, cframe= white 1pt 0in 0.0975in]%
  %%   {raw/distance_rewire_L3.pdf}
  %%   %% \put(\xin,\yin){\fboxsep=3pt\colorbox{white}{E}}
  %% \end{overpic}

  %% \begin{overpic}[width=\w-0.21in, cframe= white 1pt 0in 0.0975in]%
  %%   {raw/distance_rewire_L4.pdf}
  %%   %% \put(\xin,\yin){\fboxsep=3pt\colorbox{white}{F}}
  %% \end{overpic}

  %% \begin{overpic}[width=\w]%
  %%   {raw/0b6c065b.pdf}
  %%   %% \put(\xin,\yin){\fboxsep=3pt\colorbox{white}{G}}
  %% \end{overpic}
  
  %% \begin{overpic}[width=\w]%
  %%   {raw/ecc48ba4.pdf}
  %%   %% \put(\xin,\yin){\fboxsep=3pt\colorbox{white}{wrong plots!}}
  %% \end{overpic}

\end{tabular}	


\end{document}
